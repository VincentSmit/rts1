\begin{homeworkProblem}[Literature study on buffers]
\label{sec:lid_stud} 

\begin{homeworkSection}{FIFO buffers}
	First-in-first-out (FIFO) is a simple concept where tasks are executed in order of their arrival. In other words, it is a queueing technique where the next executed task is the task that has been in the queue the longest. 
	FIFO is also known as first-come-first-serve (FCFS).
	In practice, these buffers are often implemented as circular buffer. 
\end{homeworkSection}

\begin{homeworkSection}{Circular buffers}
	In a circular buffer, data is written into the buffer in a ring like structure. The buffer keeps track of where a data structure has been written last, but also from what location is read last. 
	On the next read or write action, data is removed from or written to the specified location. 
	When this action is complete, the last read or write indicator is updated. 
	Once the end of the buffer has been reached, the indicator wraps around and starts again at beginning of the buffer. It thus overwrites the data previously stored there.

	\todo{Evaluation criteria}

	A method for comparing circular buffer implementations involves the required overhead. Implementations with both two and four variables for bookkeeping exist. \cite{buffers} 

	The first alternative with two variables uses a pointer and a counter. The counter keeps track of the amount of tokens in the buffer, whereas the pointer represents for exmple the start index of the tokens. 
	In this case the producer and consumer both need to update the counter. This requires an atomic increment and decrement operation, which is a small downside. 

	The alternative with two variables utilizes two pointers, being a read pointer and a write pointer. 
	The read pointer indicates the first full token whereas the write pointer indicates the first empty token. 
	The producer updates write pointer and the consumer updates the read pointer. 
	Because the producer and consumer do not share a pointer, the update operations need not be atomic. The downside of this implementation 
	the case when the two pointers point to the same location. Without proper bookkeeping, it is unclear whether te buffer is empty or full when this event occurs. 
	Wrap-around counters can fix this problem, but they require extra overhead. Another solution is to disallow the write pointer to catch up with the read pointer. 
	This means that when the pointers are equal, the buffer is empty.

	Besides the implementations discussed above, alternatives with four control variables exist. 
	Besides the pointer, both the consumer and the producer keep track of the amount of tokens they processed. 
	The difference between those numbers is equal to the amount of tokens in the channel.
\end{homeworkSection}
\end{homeworkProblem}

