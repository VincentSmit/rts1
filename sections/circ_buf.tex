\begin{homeworkProblem}[Circular buffer on dual-core ARM]
\label{sec:circ_buf}
The circular buffer is implemented with the bakery's algorithm. This algorithm is an analogy to people waiting to be served at the bakery. 

When a new customer comes in, he draws a number.
Then starts waiting for his turn. When the baker calls his number, the lowest in the line of waiting customers, he will be served. 
When the customer is done with his business with the baker, the customer leaves, discards its number and the baker calls the now lowest number. 

The algorithm works in the same manner. When a task requests a critical operation, it receives a number. This number 1 higher than the current highest number of all tasks waiting for that lock. 
After receiving a number, the tasks waits untill its number is the lowest of all waiting tasks for taht lock. 
When it finally holds the lowest number, the tasks executes its critical operation and leaves afterwards. When leaving, its number is \'discarded\' by a reset to 0. 
This process repeats everytime a task tries to enter a critical section.


\end{homeworkProblem}