\documentclass{article}

\usepackage{fancyhdr} % Required for custom headers
\usepackage{lastpage} % Required to determine the last page for the footer
\usepackage{extramarks} % Required for headers and footers
\usepackage[usenames,dvipsnames]{color} % Required for custom colors
\usepackage{graphicx} % Required to insert images
\usepackage{listings} % Required for insertion of code
\usepackage{courier} % Required for the courier font
\usepackage{lipsum} % Used for inserting dummy 'Lorem ipsum' text into the template
\usepackage{subcaption}
\usepackage{multirow}
\usepackage{enumitem}
\usepackage{amsmath}
\usepackage{xspace}

% Margins
\topmargin=-0.45in
\evensidemargin=0in
\oddsidemargin=0in
\textwidth=6.5in
\textheight=9.0in
\headsep=0.25in

\linespread{1.1} % Line spacing

% Set up the header and footer
\pagestyle{fancy}
 % Top right header
\lfoot{\lastxmark} % Bottom left footer
\cfoot{} % Bottom center footer
\rfoot{Page\ \thepage\ of\ \protect\pageref{LastPage}} % Bottom right footer
\renewcommand\headrulewidth{0.4pt} % Size of the header rule
\renewcommand\footrulewidth{0.4pt} % Size of the footer rule

\setlength\parindent{0pt} % Removes all indentation from paragraphs

%----------------------------------------------------------------------------------------
%	CODE INCLUSION CONFIGURATION
%----------------------------------------------------------------------------------------

\definecolor{MyDarkGreen}{rgb}{0.0,0.4,0.0} % This is the color used for comments

\lstloadlanguages{vhdl} % Load Perl syntax for listings, for a list of other languages supported see: ftp://ftp.tex.ac.uk/tex-archive/macros/latex/contrib/listings/listings.pdf

\lstloadlanguages{C} % Load C syntax for listings, for a list of other languages supported see: ftp://ftp.tex.ac.uk/tex-archive/macros/latex/contrib/listings/listings.pdf

\lstloadlanguages{tcl}


\lstloadlanguages{Haskell}

\lstset{ 
        frame=single, % Single frame around code
        basicstyle=\small\ttfamily, % Use small true type font
        keywordstyle=[1]\color{Blue}\bf, % Perl functions bold and blue
        keywordstyle=[2]\color{Purple}, % Perl function arguments purple
        keywordstyle=[3]\color{Blue}\underbar, % Custom functions underlined and blue
        identifierstyle=, % Nothing special about identifiers                                         
        commentstyle=\usefont{T1}{pcr}{m}{sl}\color{MyDarkGreen}\small, % Comments small dark green courier font
        stringstyle=\color{Purple}, % Strings are purple
        showstringspaces=false, % Don't put marks in string spaces
        tabsize=5, % 5 spaces per tab
        %
        % Put standard VHDL functions not included in the default language here
        morekeywords={    library,use,all,entity,is,port,in,out,end,architecture,of,
    begin,and},
        %
        % Put Perl function parameters here
        morekeywords=[2]{on, off, interp},
        %
        % Put user defined functions here
        morekeywords=[3]{test},
       	%
        morecomment=[l][\color{Blue}]{...}, % Line continuation (...) like blue comment
        numbers=left, % Line numbers on left
        firstnumber=1, % Line numbers start with line 1
        numberstyle=\tiny\color{Blue}, % Line numbers are blue and small
        stepnumber=5 % Line numbers go in steps of 5
}

% Creates a new command to include a vhdl script, the first parameter is the filename of the script (without .pl), the second parameter is the caption
\newcommand{\vhdlscript}[2]{
\begin{itemize}
\item[]\lstinputlisting[language=vhdl,caption=#2,label=#1]{../code/#1.vhd}
\end{itemize}
}

\newcommand{\haskellscript}[2]{
\begin{itemize}
\item[]\lstinputlisting[language=Haskell,caption=#2,label=#1]{../code/#1.hs}
\end{itemize}
}

% Creates a new command to include a perl script, the first parameter is the filename of the script (without .pl), the second parameter is the caption
\newcommand{\Cscript}[2]{
\begin{itemize}
\item[]\lstinputlisting[language=C,caption=#2,label=#1]{../code/#1.c}
\end{itemize}
}

\newcommand{\TXTscript}[2]{
\begin{itemize}
\item[]\lstinputlisting[language={},caption=#2,label=#1]{../code/#1.txt}
\end{itemize}
}

\newcommand{\TCLscript}[2]{
\begin{itemize}
\item[]\lstinputlisting[language=tcl,caption=#2,label=#1]{../code/#1.tcl}
\end{itemize}
}

\newcommand{\todo}[1]{\emph{\color{red}TODO: #1}}

%----------------------------------------------------------------------------------------
%	DOCUMENT STRUCTURE COMMANDS
%	Skip this unless you know what you're doing
%----------------------------------------------------------------------------------------

% Header and footer for when a page split occurs within a problem environment
\newcommand{\enterProblemHeader}[1]{
\nobreak\extramarks{#1}{#1 continued on next page\ldots}\nobreak
\nobreak\extramarks{#1 (continued)}{#1 continued on next page\ldots}\nobreak
}

% Header and footer for when a page split occurs between problem environments
\newcommand{\exitProblemHeader}[1]{
\nobreak\extramarks{#1 (continued)}{#1 continued on next page\ldots}\nobreak
\nobreak\extramarks{#1}{}\nobreak
}

%\setcounter{secnumdepth}{0} % Removes default section numbers
\newcounter{homeworkProblemCounter} % Creates a counter to keep track of the number of problems

\newcommand{\homeworkProblemName}{}
\newenvironment{homeworkProblem}[1][Exercise \arabic{homeworkProblemCounter}]{ % Makes a new environment called homeworkProblem which takes 1 argument (custom name) but the default is "Problem #"
\stepcounter{homeworkProblemCounter} % Increase counter for number of problems
\renewcommand{\homeworkProblemName}{#1} % Assign \homeworkProblemName the name of the problem
\section{\homeworkProblemName} % Make a section in the document with the custom problem count
\enterProblemHeader{\homeworkProblemName} % Header and footer within the environment
}{
\exitProblemHeader{\homeworkProblemName} % Header and footer after the environment
}

\newcommand{\problemAnswer}[1]{ % Defines the problem answer command with the content as the only argument
\noindent\framebox[\columnwidth][c]{\begin{minipage}{0.98\columnwidth}#1\end{minipage}} % Makes the box around the problem answer and puts the content inside
}

\newcommand{\homeworkSectionName}{}
\newenvironment{homeworkSection}[1]{ % New environment for sections within homework problems, takes 1 argument - the name of the section
\renewcommand{\homeworkSectionName}{#1} % Assign \homeworkSectionName to the name of the section from the environment argument
\subsection{\homeworkSectionName} % Make a subsection with the custom name of the subsection
\enterProblemHeader{\homeworkProblemName\ [\homeworkSectionName]} % Header and footer within the environment
}{
\enterProblemHeader{\homeworkProblemName} % Header and footer after the environment
}

%----------------------------------------------------------------------------------------
%	NAME AND CLASS SECTION
%----------------------------------------------------------------------------------------

\newcommand{\hmwkTitle}{} % Assignment title
\newcommand{\hmwkDueDate}{Thursday,\ April\ 21,\ 2016} % Due date
\newcommand{\hmwkClass}{Real-time Systems 1} % Course/class
\newcommand{\hmwkClassInstructor}{S. Gerez} % Teacher/lecturer
\newcommand{\hmwkAuthorName}{Stephen Geerlings, s???????, EMSYS} % Your name
\newcommand{\hmwkAuthorNameTwo}{V.J. Smit, s1206257, EMSYS} % Your name

%----------------------------------------------------------------------------------------
%	TITLE PAGE
%----------------------------------------------------------------------------------------

\title{
\vspace{2in}
\textmd{\textbf{\hmwkClass:\ \hmwkTitle}}\\
\normalsize\vspace{0.1in}\small{\hmwkDueDate}\\
\vspace{0.1in}\large{\textit{\hmwkClassInstructor\ }}
\vspace{3in}
}
	
\author{\textbf{\hmwkAuthorName}\\ \textbf{\hmwkAuthorNameTwo}}

\date{} % Insert date here if you want it to appear below your name

%----------------------------------------------------------------------------------------

\begin{document}

\maketitle

%----------------------------------------------------------------------------------------
%	TABLE OF CONTENTS
%----------------------------------------------------------------------------------------

%\setcounter{tocdepth}{1} % Uncomment this line if you don't want subsections listed in the ToC

\newpage
\tableofcontents
\newpage

\begin{homeworkProblem}[Introduction]
\label{sec:introduction} 
	
\end{homeworkProblem}



\begin{homeworkProblem}[Literature study on buffers]
\label{sec:lid_stud} 

\begin{homeworkSection}{FIFO buffers}
	First-in-first-out (FIFO) is a simple concept where tasks are executed in order of their arrival. In other words, it is a queueing technique where the next executed task is the task that has been in the queue the longest. 
	FIFO is also known as first-come-first-serve (FCFS).
	In practice, these buffers are often implemented as circular buffer. Especially when the data is no longer relevant after it has been read.
\end{homeworkSection}

\begin{homeworkSection}{Circular buffers}
	In a circular buffer, data is written into the buffer in a ring like structure. The buffer keeps track of where a data structure has been written last, but also from what location is read last. 
	On the next read or write action, data is removed from or written to the specified location. 
	When this action is complete, the last read or write indicator is updated. 
	Once the end of the buffer has been reached, the indicator wraps around and starts again at beginning of the buffer. It thus overwrites the data previously stored there.

\subsubsection{Evaluation criteria}
	In order to evaluate a buffer implementation certain criteria need to be specified. Multiple features of a buffer are suitable for evaluating an implementation.
	
	A method for comparing circular buffer implementations involves the required overhead. Implementations with both two and four variables for bookkeeping exist. \cite{buffers} 
	Implementations with more variables have more control over the state of the system system. On the other hand, more control introduces more overhead. This might cost more time and is more errorprone.

	Since both read and write operations will occur, the extend to which synchronization is required is a subject for evaluation. The need for atomic operations is obvious, but some implementations need more of them than others.

	Besides, complexity is often an important performance measure. This is not the case with these buffer implementations, though. There is always a direct indicator of where to read or write. This feature makes the complexity of $O(1)$.

	Last, but maybe most important, it is important that the buffer implements all synchronization correctly. This is vital to prevent deadlocks, race conditions etcetera.
	
\subsubsection{Implementations}
	The first alternative with two variables uses a pointer and a counter. The counter keeps track of the amount of tokens in the buffer, whereas the pointer represents for exmple the start index of the tokens. 
	In this case the producer and consumer both need to update the counter. This requires an atomic increment and decrement operation. 
	This is a slight disadvantage, since it might have a poor effect on the performance of the system.
	
	The alternative with two variables utilizes two pointers, being a read pointer and a write pointer. 
	The read pointer indicates the first full token whereas the write pointer indicates the first empty token. 
	The producer updates write pointer and the consumer updates the read pointer. 
	Because the producer and consumer do not share a pointer, the update operations need not be atomic. The downside of this implementation 
	the case when the two pointers point to the same location. Without proper bookkeeping, it is unclear whether te buffer is empty or full when this event occurs. 
	Wrap-around counters can fix this problem, but they require extra overhead. Another solution is to disallow the write pointer to catch up with the read pointer. 
	This means that when the pointers are equal, the buffer is empty.

	Besides the implementations discussed above, alternatives with four control variables exist. 
	Besides the pointer, both the consumer and the producer keep track of the amount of tokens they processed. 
	The difference between those numbers is equal to the amount of tokens in the channel.
\end{homeworkSection}
\end{homeworkProblem}



\begin{homeworkProblem}[Circular buffer on dual-core ARM]
\label{sec:circ_buf}
The methods of api.c are implemented in such a way that the api offers the user access to a circular FIFO buffer. The api keeps a list of all the opened buffers and provides methods for reading and writing.

Per buffer we keep track of the current write and read pointers and whether we have a full or empty buffer. This is necessary as it is unclear whether a buffer is full or empty when the write and read pointers are the same.

The circular buffer resource access management is implemented with the bakery algorithm. This algorithm is an analogy to people waiting to be served at a bakery. We started with using regular spinlocks for access. This approach led to starvation of tasks. We noticed the hints in the commentary of api.c and decided that a more advanced approach was necessary. This led to our implementation of the bakery algorithm.

When a new customer comes in, he draws a number.
Then starts waiting for his turn. When the baker calls his number, the lowest in the line of waiting customers, he will be served. 
When the customer is done with his business with the baker, the customer leaves, discards its number and the baker calls the now lowest number. 

The algorithm works in the same manner. When a task requests a critical operation, it receives a number. This number 1 higher than the current highest number of all tasks waiting for that lock. 
After receiving a number, the tasks waits untill its number is the lowest of all waiting tasks for that lock. 
When it finally holds the lowest number, the tasks executes its critical operation and leaves afterwards. When leaving, its number is \'discarded\' by a reset to 0. 
This process repeats everytime a task tries to enter a critical section.

In listing~\ref{api} is the implementation shown of the bakery algoritm and the full api. The file was mostly implemented as proof of concept and tested often. Possible improvements of the implementation are removal of magic numbers, better commentary and proper use of structs to manage data. The api should also offer some kind of error codes and standardized return values. However, these aren't used in the test1.c file and often we had to guess what the actual use of a method was. 

Test of the implementation led to some difficulties. The first is the unblocking nature of printf. We often expected output that never came due to the high data output of the test. Our simple spin lock implementation worked properly and the buffers were protected from overflow. The bakery implementation unfortunately deadlocks after the first read from the buffer done by task 2, as can be seen in listing~\ref{api_out}. We have tried debugging the bakery algorithm but were unable to improve the performance.

\Cscript{api}{Implementation of api.c}

\newpage

\TXTscript{api_out}{Output of the SWARM simulator.}

\end{homeworkProblem}

\begin{homeworkProblem}[HAPI simulator exercises]
\label{sec:hapi}

\begin{homeworkSection}{Exercise 1}
	
\end{homeworkSection}

\begin{homeworkSection}{Exercise 2}
	
\end{homeworkSection}

\end{homeworkProblem}

\section{References} % (fold)
\label{sec:references}

% section references (end)
\begin{thebibliography}{1}

\bibitem{buffers}
	\emph{Design and programming of embedded multiprocessors: An interface-centric approach},
  	P. van der Wolf, E. de Kock, T. Henriksson, W. Kruijtzer, and G. Essink.

\bibitem{algorithms}
	\emph{Introduction to algorithms},
	T. Cormen, C. Leiserson, R. Rivest, C. Stein.

\end{thebibliography}

\end{document}